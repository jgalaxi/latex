\documentclass[12pt,a4paper]{article}

% Paquetes de configuración del documento
\usepackage[utf8]{inputenc}
\usepackage[spanish]{babel}
\usepackage[T1]{fontenc}
\usepackage[margin=2.5cm]{geometry}
\usepackage{fancyhdr}
%Paquetes para simbologia%
\usepackage{amsmath}
\usepackage{amsfonts}
\usepackage{amssymb}
\usepackage{physics}
\usepackage{longtable}
\usepackage{graphicx}
\usepackage{tabularx}
\usepackage{caption}
\usepackage{float}
\usepackage[hyphens]{url}        % Permite cortar URLs largas con guiones
\usepackage[colorlinks=true,
            linkcolor=black,
            urlcolor=myblue,
            citecolor=black,
            filecolor=black]{hyperref}
 % Opción estándar para enlaces
\Urlmuskip=0mu plus 1mu          % Mejora el espaciado para permitir cortes
\usepackage{subcaption}  % en el preámbulo
\usepackage{pgfplots}
\pgfplotsset{compat=1.18}
\usepackage{tikz}
\usepackage{xcolor}
\definecolor{myblue}{RGB}{42, 127, 179}

\pagestyle{fancy}
\chead{\textit{Materiales Metálicos}}
\rhead{\textit{UTN-FRVM}}
\lhead{\textit{Ingeniería Mecánica}}

\begin{document}
\begin{titlepage}
	
	\begin{center}
		{\huge \textit{Universidad Tecnológica Nacional}}\\
        \vspace{0.5cm}
		{\LARGE \textit{Facultad Regional Villa María}}\\
		\vspace{1.5cm}
        {\LARGE{\textit{Ingeniería Mecánica - Materiales Metálicos}}}\\
		\vspace{1.5cm}
        \LARGE{\textit{Trabajo Práctico 3-06}}
	\end{center}
	
	\vfill

    \textit{Grupo DEL RÍO:}
	\begin{itemize}
		\item \textit{Abregú, Iván.}
		\item \textit{Antico, Rodrigo.}
		\item \textit{Brussa,Julián.}
		\item \textit{Cabral, Franco.}
        \item \textit{Cárdenas, Felipe.}
        \item \textit{Cardozo, Martín.}
        \item \textit{Córdoba, Nathan.}
        \item \textit{Cucco, Ramiro.}
        \item \textit{del Río, Juan.}
        \item \textit{Guerini, Nazareno.}
        \item \textit{Medina, Ivo.}
        \item \textit{Ortiz, Gastón.}
        \item \textit{Picos, Elías.}
        \item \textit{Quinteros, Lautaro.}
	\end{itemize}
    
	\textit{Docentes:}
	\begin{itemize}
		\item \textit{Dr. Lucioni, Eldo José.}
		\item \textit{Ing. Victorio Vallaro, Juan Manuel.}
	\end{itemize}
	\centering
	\today
	
\end{titlepage}

\newpage
\tableofcontents

\begin{abstract}
    Investigue los métodos de obtención de arrabio, acero y fundición a fin de adquirir la capacidad de explicar conceptualmente los mismos. La actividad requerida incluye la identificación del tipo y uso de los hornos asociados a dichos métodos de obtención. [NOTA: A modo de orientación, puede consultar la siguiente fuente de información:]
    \begin{itemize}
        \item Aguilar Schafer, J.A. Yacimientos minerales y procesos geológicos. Sitio Web: biblio3.
        \item Aguilar Schafer, J.A. Explotación minera, preparación y concentración. Sitio Web: biblio3.
        \item Aguilar Schafer, J.A. Metalurgia extractiva del hierro. Sitio Web: biblio3.u.
        \item Aguilar Schafer, J.A. Hornos Industriales. Sitio Web: biblio3.ur.
    \end{itemize}
\end{abstract}

\section{Obtención de Arrabio (Pig Iron)}

El arrabio es el producto primario de la reducción del mineral de hierro, con un alto contenido de carbono (alrededor del 4\%) y otras impurezas como silicio, manganeso, fósforo y azufre. Se obtiene mediante un proceso de reducción indirecta y directa en un horno alto, donde el mineral de hierro (como hematita o magnetita) se reduce usando coque como agente reductor y caliza como fundente para eliminar impurezas en forma de escoria.

\subsection{Método conceptual}

\begin{itemize}
    \item \textbf{Carga:} Mineral de hierro (2 t), coque (1 t), caliza (0.5 t) y aire (4 t).
    \item \textbf{Reacciones principales:}
    \begin{itemize}
        \item Reducción indirecta (en la zona superior): $3\mathrm{Fe}_2\mathrm{O}_3 + \mathrm{CO} \to 2\mathrm{Fe}_3\mathrm{O}_4 + \mathrm{CO}_2$; $\mathrm{Fe}_3\mathrm{O}_4 + \mathrm{CO} \to 3\mathrm{FeO} + \mathrm{CO}_2$; $\mathrm{FeO} + \mathrm{CO} \to \mathrm{Fe} + \mathrm{CO}_2$.
        \item Reducción directa (en la zona media): $\mathrm{FeO} + \mathrm{C} \to \mathrm{Fe} + \mathrm{CO}$.
        \item Fusión y carburación: $3\mathrm{Fe} + 2\mathrm{CO} \to \mathrm{Fe}_3\mathrm{C} + \mathrm{CO}_2$.
        \item Desulfuración y formación de escoria: $\mathrm{CaCO}_3 \to \mathrm{CaO} + \mathrm{CO}_2$; impurezas como $\mathrm{SiO}_2$ se combinan con $\mathrm{CaO}$ para formar escoria.
    \end{itemize}
    \item \textbf{Producto:} Arrabio (1 t, con composición típica: Fe 93.7\%, C 4.5\%, Mn 0.4\%, Si 0.45\%, P 0.11\%, S 0.025\%), escoria (0.5 t) y gases (6 t).
    \item \textbf{Temperaturas:} Desde 200°C en la parte superior hasta 1500°C en la zona de fusión.
    \item \textbf{Eficiencia:} Se mejora con aire precalentado (1030°C) y enriquecido en oxígeno, reduciendo pérdidas térmicas.
\end{itemize}

\subsection{Horno asociado}

\begin{itemize}
    \item \textbf{Tipo:} Alto Horno (Blast Furnace).
    \item \textbf{Uso:} Reducción de minerales de hierro en estado sólido para producir arrabio líquido. Es un horno continuo de gran capacidad (hasta 5000 t/día), con estructura cilíndrica refractaria, toberas para inyección de aire caliente y regeneradores para precalentar el aire.
    \item \textbf{Características:} Altura de 30-40 m, funciona a contracorriente (carga por arriba, gases por abajo). Asociado a procesos como sinterización de minerales y producción de coque.
\end{itemize}
\begin{table}[h]
    \centering
    \caption{Carga típica en Alto Horno de CSH}
    \begin{tabular}{|l|r|r|}
        \hline
        Componente & Kg/t & Descripción \\ \hline
        Mineral de hierro & 490 & Fuente principal de Fe. \\
        Pellets & 995 & Mineral aglomerado. \\
        Chatarra & 15 & 300 \\
        Mineral de Mn & 22 & 450 \\
        Caliza & 112 & Fundente para escoria. \\
        Cuarzo & 12 & 250 \\
        Coque & 451 & Reductor y combustible. \\
        Petróleo + Alquitrán & 44 & 899 \\
        Aire Insuflado & 1530 m³/min & Oxidante precalentado a 1030°C. \\
        \hline
    \end{tabular}
\end{table}
Composición del arrabio: Fe 93.7\%, C 4.5\%, Mn 0.4\%, Si 0.45\%, P 0.11\%, S 0.025\%.

\section{Obtención de Acero (Steel)}

El acero se obtiene a partir del arrabio mediante procesos de afino, donde se reduce el carbono (a 0.05-1.5\%) y se eliminan impurezas (Si, Mn, P, S) mediante oxidación. Hay métodos con oxígeno (convertidores) y eléctricos. Incluye fases de oxidación (eliminar C e impurezas) y reducción (eliminar S y óxidos).

\subsection{Método conceptual}

\begin{itemize}
    \item \textbf{Carga típica:} Arrabio (93\% Fe, 4\% C, 0.5-2\% Si, 1\% Mn, 2-0.1\% P, 0.05\% S) + chatarra + fundentes.
    \item \textbf{Reacciones principales:}
    \begin{itemize}
        \item Oxidación: $\mathrm{Si} + \mathrm{O}_2 \to \mathrm{SiO}_2$; $\mathrm{Mn} + \frac{1}{2}\mathrm{O}_2 \to \mathrm{MnO}$; $2\mathrm{C} + \mathrm{O}_2 \to 2\mathrm{CO}$; $\mathrm{P} + \frac{5}{2}\mathrm{O}_2 + \frac{3}{2}\mathrm{CaO} \to \frac{1}{2}\mathrm{Ca}_3(\mathrm{PO}_4)_2$.
        \item Desulfuración: $\mathrm{FeS} + \mathrm{CaO} \to \mathrm{Fe} + \mathrm{CaS}$.
        \item Adiciones: Desoxidantes (Al, Si) y ferroaleaciones para ajustar composición.
    \end{itemize}
    \item \textbf{Metalurgia secundaria:} Agitación (con argón o EMS), desgasificación (RH o tanque), horno cuchara para recalentamiento y ajustes.
    \item \textbf{Producto:} Acero con Fe 98\%, C 0.05-1.5\%, Si 0.5-2\%, Mn 0.3-0.6\%, P/S <0.05\%.
\end{itemize}

\subsection{Hornos asociados}

\begin{table}[h]
    \centering
    \caption{Comparación de Hornos para Acero}
    \resizebox{\textwidth}{!}{
    \begin{tabular}{|l|c|c|c|c|}
        \hline
        Horno & Tipo de Energía & Uso Principal & Ventajas & Desventajas \\ \hline
        Bessemer/Thomas & Aire & Afino rápido de arrabio & Económico, simple & Obsoleto, alto P/S en producto \\
        LD/BOF & Oxígeno puro & Producción masiva de acero & Alta eficiencia, bajo costo & Requiere arrabio puro \\
        Siemens-Martin & Gas/combustibles & Afino con chatarra & Flexible para aleaciones & Alto consumo energético \\
        Arco Eléctrico & Eléctrica & Reciclaje de chatarra & Bajo impacto ambiental, preciso & Alto costo eléctrico \\
        Inducción & Eléctrica & Aceros especiales & Uniformidad, sin contaminación & Capacidad limitada \\ \hline
    \end{tabular}
    }
\end{table}

\section{Obtención de Fundición (Cast Iron)}

La fundición o hierro colado se obtiene remoldeando arrabio y chatarra, con alto carbono (2.5-3.75\%) para propiedades de fundición. Es gris (grafito libre) o blanca (cementita).

\subsection{Método conceptual}

\begin{itemize}
    \item \textbf{Carga:} Arrabio, chatarra, coque y fundentes.
    \item \textbf{Reacciones:} Fusión y ajuste de C/Si para formar grafito o cementita. Reducción de S/P.
    \item \textbf{Producto:} Fundición gris (maleable, usada en piezas fundidas) o blanca (dura, para laminación).
\end{itemize}

\subsection{Horno asociado}

\begin{itemize}
    \item \textbf{Tipo:} Horno de Cubilote (Cupola Furnace).
    \item \textbf{Uso:} Fusión de arrabio y chatarra con coque y aire. Es un horno vertical con toberas, para producción continua de fundición (capacidad 10-50 t/h). Usado en fundiciones para piezas como bloques de motor o tuberías.
    \item \textbf{Características:} Contacto directo entre combustible, material y productos de combustión. Bajo costo, pero alto consumo de coque.
\end{itemize}

\begin{table}
    \caption{Tipos de Fundición}
     \resizebox{\textwidth}{!}{
    \begin{tabular}{|l|c|c|c|}
        \hline
        Tipo de Fundición & Contenido de C & Horno & Uso \\ \hline
        Fundición Gris & 2.5-3.75\% & Cubilote & Piezas fundidas maleables (e.g., motores) \\
        Fundición Blanca & 2-4\% & Cubilote o Inducción & Material duro para laminación o aleaciones \\ \hline
    \end{tabular}}
\end{table}

Estos métodos permiten explicar la transformación del mineral de hierro en productos útiles, con hornos diseñados para eficiencia energética y control de impurezas. El alto horno es clave para arrabio, convertidores y eléctricos para acero, y cubilote para fundición.

\end{document}