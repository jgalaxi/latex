\documentclass[12pt,a4paper]{article}

% Paquetes de configuración del documento
\usepackage[utf8]{inputenc}
\usepackage[spanish]{babel}
\usepackage[T1]{fontenc}
\usepackage[margin=2.5cm]{geometry}
\usepackage{fancyhdr}
%Paquetes para simbologia%
\usepackage{amsmath}
\usepackage{amsfonts}
\usepackage{amssymb}
\usepackage{physics}
\usepackage{longtable}
\usepackage{graphicx}
\usepackage{caption}
\usepackage{float}
\usepackage{xurl}
\usepackage{siunitx}
\usepackage[colorlinks=true,
            linkcolor=black,
            urlcolor=myblue,
            citecolor=black,
            filecolor=black]{hyperref}
 % Opción estándar para enlaces
\Urlmuskip=0mu plus 1mu          % Mejora el espaciado para permitir cortes
\usepackage{subcaption}  % en el preámbulo
\usepackage{pgfplots}
\pgfplotsset{compat=1.18}
\usepackage{tikz}
\usepackage{xcolor}
\definecolor{myblue}{RGB}{42, 127, 179}

\pagestyle{fancy}
\chead{\textit{Materiales Metálicos}}
\rhead{\textit{UTN-FRVM}}
\lhead{\textit{Ingeniería Mecánica}}

\begin{document}
\begin{titlepage}
	
	\begin{center}
		{\huge \textit{Universidad Tecnológica Nacional}}\\
        \vspace{0.5cm}
		{\LARGE \textit{Facultad Regional Villa María}}\\
		\vspace{1.5cm}
        {\LARGE{\textit{Ingeniería Mecánica - Materiales Metálicos}}}\\
		\vspace{1.5cm}
        \LARGE{\textit{Trabajo Práctico 3-05}}
	\end{center}
	
	\vfill

    \textit{Grupo DEL RÍO:}
	\begin{itemize}
		\item \textit{Abregú, Iván.}
		\item \textit{Antico, Rodrigo.}
		\item \textit{Brussa,Julián.}
		\item \textit{Cabral, Franco.}
        \item \textit{Cárdenas, Felipe.}
        \item \textit{Cardozo, Martín.}
        \item \textit{Córdoba, Nathan.}
        \item \textit{Cucco, Ramiro.}
        \item \textit{del Río, Juan.}
        \item \textit{Guerini, Nazareno.}
        \item \textit{Medina, Ivo.}
        \item \textit{Ortiz, Gastón.}
        \item \textit{Picos, Elías.}
        \item \textit{Quinteros, Lautaro.}
	\end{itemize}
    
	\textit{Docentes:}
	\begin{itemize}
		\item \textit{Dr. Lucioni, Eldo José.}
		\item \textit{Ing. Victorio Vallaro, Juan Manuel.}
	\end{itemize}
	\centering
	\today
	
\end{titlepage}

\newpage
\tableofcontents

\begin{abstract}
    Propiedades mecánicas de alambres e hilos (filamentos).
    \begin{itemize}
        \item Determinar el módulo de elasticidad, tensión de fluencia, tensión máxima, tensión de rotura, resiliencia y tenacidad de un alambre de material ferroso y de un alambre de material no ferroso. (Alambre: sección > 1 mm2).
        \item Determinar la pendiente de la curva de termofluencia para un hilo (filamento) de material ferroso y un material no ferroso. (Hilo: sección < 1 mm2).
        \item Verificar y contrastar los resultados obtenidos con la bibliografía de referencia (Ej: normas, libros, catálogos, etc.).
        \item CONDICIÓN: Para la realización de los ensayos deberán emplearse máquinas, dispositivos o equipos diseñados y construidos por los integrantes de cada equipo. No podrán emplearse máquinas, dispositivos o equipos comerciales.
    \end{itemize}
\end{abstract}

\section{Introduccion.}

En el presente trabajo se realizaron ensayos sobre hilo y alambre de acero 1045 y sobre alambre de aluminio, para poder medir, graficar y comparar las propiedades mecanicas de los diferentes materiales y la diferencias que existen entre materiales ferrosos y no ferrosos.


\section{Metodo y herramientas/maquinas.}

\subsection{ensayo de termofluencia:}
Para poder realizar el ensayo de termofluenciase utilizo una pistola de calor capaz de alcanzar temperaturas entre \SI{450}{\celsius} y \SI{500}{\celsius}, unas pesas de entre \SI{5}{\kilogram} y \SI{10}{\kilogram}, unidas entre si para poder alcanzar los \SI{30}{\kilogram}, un soporte y unas agarraderas para poder mantener fijo el alambre en el momento del ensayo junto con una barra con marcas de longuitud al lado de la probeta para poder medir la deformacion.

\subsection{ensayo de traccion:}
Para este ensayo se utilizo un dinamometro de \SI{50}{\kilogram} de fuerza maxima, y originalmente se iba a utilizar una maquina de traccion armada por un grupo de alumnos en años pasados, pero al utilizarla no lograbamos romper el alambre con la fuerza de \SI{40}{\kilogram} (No mas peso que eso para asegurarnos de no romper el dinamometro), por lo que se decidio construir una maquina nueva que funciona con un brazo de palanca, asi la fuerza que se le hace a la probeta es mayor, que la que esta recibiendo el dinamometro.

Y esto se ve en la 
\begin{equation}
    F_1\cdot d_1 = F_2\cdot d_2
\end{equation}
Donde:
\begin{itemize}
    \item $F_1$ es la fuerza que se le esta haciendo al dinamometro.
    \item $F_2$ es la fuerza que se le esta haciendo a la probeta.
    \item $d_1$ es la distancia del dinamometro al punto de apoyo.
    \item $d_2$ es la distancia de la probeta al punto de apoyo 
\end{itemize}

Y podemos obtener la fuerza que se le hace a la probeta despejando $F_2$.

\begin{equation}
    \frac{F_1\cdot d_1}{d_2} = F_2
\end{equation}

Y asi mientras sepamos las dos dintancias al punto de apoyo y la fuerza que nos marca el dinamometro podemos sacar la fuerza ejercida a la probeta.


\section{resultados}

\end{document}