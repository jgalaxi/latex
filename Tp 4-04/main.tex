\documentclass[12pt,a4paper]{article}

% Paquetes de configuración del documento
\usepackage[utf8]{inputenc}
\usepackage[spanish]{babel}
\usepackage[T1]{fontenc}
\usepackage[margin=2.5cm]{geometry}
\usepackage{fancyhdr}
%Paquetes para simbologia%
\usepackage{amsmath}
\usepackage{amsfonts}
\usepackage{amssymb}
\usepackage{physics}
\usepackage{longtable}
\usepackage{graphicx}
\usepackage{caption}
\usepackage{float}
\usepackage{xurl}
\usepackage[colorlinks=true,
            linkcolor=black,
            urlcolor=myblue,
            citecolor=black,
            filecolor=black]{hyperref}
 % Opción estándar para enlaces
\Urlmuskip=0mu plus 1mu          % Mejora el espaciado para permitir cortes
\usepackage{subcaption}  % en el preámbulo
\usepackage{pgfplots}
\pgfplotsset{compat=1.18}
\usepackage{tikz}
\usepackage{xcolor}
\definecolor{myblue}{RGB}{42, 127, 179}

\pagestyle{fancy}
\chead{\textit{Materiales Metálicos}}
\rhead{\textit{UTN-FRVM}}
\lhead{\textit{Ingeniería Mecánica}}

\begin{document}
\begin{titlepage}
	
	\begin{center}
		{\huge \textit{Universidad Tecnológica Nacional}}\\
        \vspace{0.5cm}
		{\LARGE \textit{Facultad Regional Villa María}}\\
		\vspace{1.5cm}
        {\LARGE{\textit{Ingeniería Mecánica - Materiales Metálicos}}}\\
		\vspace{1.5cm}
        \LARGE{\textit{Trabajo Práctico 3-06}}
	\end{center}
	
	\vfill

    \textit{Grupo DEL RÍO:}
	\begin{itemize}
		\item \textit{Abregú, Iván.}
		\item \textit{Antico, Rodrigo.}
		\item \textit{Brussa,Julián.}
		\item \textit{Cabral, Franco.}
        \item \textit{Cárdenas, Felipe.}
        \item \textit{Cardozo, Martín.}
        \item \textit{Córdoba, Nathan.}
        \item \textit{Cucco, Ramiro.}
        \item \textit{del Río, Juan.}
        \item \textit{Guerini, Nazareno.}
        \item \textit{Medina, Ivo.}
        \item \textit{Ortiz, Gastón.}
        \item \textit{Picos, Elías.}
        \item \textit{Quinteros, Lautaro.}
	\end{itemize}
    
	\textit{Docentes:}
	\begin{itemize}
		\item \textit{Dr. Lucioni, Eldo José.}
		\item \textit{Ing. Victorio Vallaro, Juan Manuel.}
	\end{itemize}
	\centering
	\today
	
\end{titlepage}

\newpage
\tableofcontents

\begin{abstract}
    A partir de la fuente de información listada a continuación, analice e investigue la Descripción, Propiedades, Características y del Tungsteno (W), el Tantalio (Ta), el Molibdeno (Mo) y la Matriz Metálica de Tungsteno (W-MMC) a fin de adquirir la capacidad de explicar el significado de la información que allí se detalla:
    \begin{itemize}
        \item Tungsteno (W). PLANSEE. Sitio Web: \url{https://www.plansee.com/es/materiales/tungsteno.html}.
        \item Tantalio (Ta). PLANSEE. Sitio Web: \url{https://www.plansee.com/es/materiales/tantalo.html}.
        \item Molibdeno (Mo). PLANSEE. Sitio Web: \url{https://www.plansee.com/es/materiales/molibdeno.html}.
        \item Matriz Metálica de Tungsteno (W-MMC). PLANSEE. Sitio Web: \url{https://www.plansee.com/es/materiales/w-mmc.html}.
    \end{itemize}
    \underline{INFORMACIÓN ADICIONAL:}
    \begin{itemize}
        \item ASM Handbook. Properties and Selection: Nonferrous Alloys and Special-Purpose Materials. ASM. Metals Handbook, 10th Edition. Vol. 2. 1992.[NOTA: Importante fuente de información donde se desarrollan variados aspectos de los materiales no ferrosos].
        \item ¿Qué tan fuerte es el titanio? ¡Prueba de presión hidráulica!. Hydraulic Press Channel Español. Sitio Web: \url{https://www.youtube.com/watch?v=CXg78qiN9rI}. [NOTA: Una interesante muestra de la resistencia a la compresión de varios materiales no ferrosos].
        \item ¿Qué tan fuertes son los metales? ¡Explosión + VENTANA ROTA! Prueba con Prensa Hidráulica. Hydraulic Press Channel Español. Sitio Web: \url{https://www.youtube.com/watch?v=bnynk21lnxg}. [NOTA 1: Una interesante muestra de la resistencia a la compresión de varios materiales no ferrosos. // NOTA 2: Prestar especial atención a la velocidad y morfología del fenómeno de rotura que se registra en la marca temporal \textcolor{red}{5’52”}].
        \item Tabla periódica | El wolframio, un elemento extremadamente denso. CienciaDeSofa. Sitio Web: \url{https://www.youtube.com/watch?v=8vZnYLLCT78&t=659s}.
        \item Tabla periódica | El COBRE, un metal que se encuentra en estado puro en la naturaleza. CienciaDeSofa. Sitio Web: \url{https://www.youtube.com/watch?v=c_Z2XMKV_6E}.
        \item Crespo Cánovas, J. Aleaciones de cobre: desarrollos recientes y nuevas perspectivas. Tesis de Grado. UPV ETSII. Año 2021. Sitio Web: \url{}.
        \item Tabla Periódica | El MAGNESIO, el SECRETO de las VELAS QUE SE ENCIENDEN SOLAS. CienciaDeSofa. Sitio Web: \url{https://www.youtube.com/watch?v=KrD4Gw4kqCY}.
        \item Tabla Periódica | El CROMO, el metal puro MÁS DURO que existe. CienciaDeSofa. Sitio Web: \url{https://www.youtube.com/watch?v=cg0RZTKzeQY}.
        \item Tabla periódica | El MOLIBDENO, un elemento que ENDURECE el acero. CienciaDeSofa. Sitio Web: \url{https://www.youtube.com/watch?v=6PPPc_1jD8M}.
        \item Tabla periódica | El ALUMINIO, el metal que FUNDE PLANETAS. CienciaDeSofa. Sitio Web: \url{https://www.youtube.com/watch?v=7pHGZqwngAk}.
        \item  TABLA PERIÓDICA | El ESTAÑO, un metal que SE DESHACE CON EL FRÍO. CienciaDeSofa. Sitio Web: \url{https://www.youtube.com/watch?v=ma9XbkMzu0U}.
        \item  ALU STOCK. Catálogos y Manuales. Sitio Web: \url{https://www.alu-stock.es/es/descargas/}. [NOTA: Información detallada sobre diversos aspectos y usos del aluminio].
        \item SANMETAL. Aluminio. Barras para mecanización y forja. Perfiles de formas regulares. Sitio Web: \url{https://www.sanmetal.es/docs/1246450322.pdf}. [NOTA: Información detallada sobre diversos aspectos y usos del aluminio].
    \end{itemize}
\end{abstract}



\end{document}