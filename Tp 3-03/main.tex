\documentclass[12pt,a4paper]{article}

% Paquetes de configuración del documento
\usepackage[utf8]{inputenc}
\usepackage[spanish]{babel}
\usepackage[T1]{fontenc}
\usepackage[margin=2.5cm]{geometry}
\usepackage{fancyhdr}
%Paquetes para simbologia%
\usepackage{amsmath}
\usepackage{amsfonts}
\usepackage{amssymb}
\usepackage{physics}
\usepackage{longtable}
\usepackage{graphicx}
\usepackage{caption}
\usepackage{float}
\usepackage{xurl}
\usepackage[colorlinks=true,
            linkcolor=black,
            urlcolor=myblue,
            citecolor=black,
            filecolor=black]{hyperref}
 % Opción estándar para enlaces
\Urlmuskip=0mu plus 1mu          % Mejora el espaciado para permitir cortes
\usepackage{subcaption}  % en el preámbulo
\usepackage{pgfplots}
\pgfplotsset{compat=1.18}
\usepackage{tikz}
\usepackage{xcolor}
\definecolor{myblue}{RGB}{42, 127, 179}

\pagestyle{fancy}
\chead{\textit{Materiales Metálicos}}
\rhead{\textit{UTN-FRVM}}
\lhead{\textit{Ingeniería Mecánica}}

\begin{document}
\begin{titlepage}
	
	\begin{center}
		{\huge \textit{Universidad Tecnológica Nacional}}\\
        \vspace{0.5cm}
		{\LARGE \textit{Facultad Regional Villa María}}\\
		\vspace{1.5cm}
        {\LARGE{\textit{Ingeniería Mecánica - Materiales Metálicos}}}\\
		\vspace{1.5cm}
        \LARGE{\textit{Trabajo Práctico 3-05}}
	\end{center}
	
	\vfill

    \textit{Grupo DEL RÍO:}
	\begin{itemize}
		\item \textit{Abregú, Iván.}
		\item \textit{Antico, Rodrigo.}
		\item \textit{Brussa,Julián.}
		\item \textit{Cabral, Franco.}
        \item \textit{Cárdenas, Felipe.}
        \item \textit{Cardozo, Martín.}
        \item \textit{Córdoba, Nathan.}
        \item \textit{Cucco, Ramiro.}
        \item \textit{del Río, Juan.}
        \item \textit{Guerini, Nazareno.}
        \item \textit{Medina, Ivo.}
        \item \textit{Ortiz, Gastón.}
        \item \textit{Picos, Elías.}
        \item \textit{Quinteros, Lautaro.}
	\end{itemize}
    
	\textit{Docentes:}
	\begin{itemize}
		\item \textit{Dr. Lucioni, Eldo José.}
		\item \textit{Ing. Victorio Vallaro, Juan Manuel.}
	\end{itemize}
	\centering
	\today
	
\end{titlepage}

\newpage
\tableofcontents

\begin{abstract}
    A partir de la bibliografía listada a continuación, analice e investigue el contenido relacionado con el acero especial indicado por la Cátedra a fin de adquirir la capacidad de explicar el significado de la información que allí se detalla. [REQUERIMIENTO ADICIONAL: Adicionalmente debe investigar si dichos materiales son de uso actual o ha sido reemplazados por otros más modernos.] [NOTA 1: Anualmente la Cátedra asignará los temas a cada equipo de trabajo // NOTA 2: Tenga en cuenta que el requerimiento adicional solo se aplica a la bibliografía indicada. Ver NOTA 3.]
    \begin{itemize}
        \item Apraiz Barreiro, J. Aceros Especiales y Otras Aleaciones. Dossat. 5ta Edición. Madrid, 1975. {MM-CAD-0.0.0}.
        \begin{itemize}
            \item Cap. X Aceros para la fabricación de chapa magnética (pp.221-248) [Caso 2025].
        \end{itemize}
    \end{itemize}
\end{abstract}


\end{document}