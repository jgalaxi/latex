\documentclass[12pt,a4paper]{article}

% Paquetes de configuración del documento
\usepackage[utf8]{inputenc}
\usepackage[spanish]{babel}
\usepackage[T1]{fontenc}
\usepackage[margin=2.5cm]{geometry}
\usepackage{fancyhdr}
%Paquetes para simbologia%
\usepackage{amsmath}
\usepackage{amsfonts}
\usepackage{amssymb}
\usepackage{physics}
\usepackage{longtable}
\usepackage{graphicx}
\usepackage{caption}
\usepackage{float}
\usepackage[hyphens]{url}        % Permite cortar URLs largas con guiones
\usepackage[colorlinks=true,
            linkcolor=black,
            urlcolor=myblue,
            citecolor=black,
            filecolor=black]{hyperref}
 % Opción estándar para enlaces
\Urlmuskip=0mu plus 1mu          % Mejora el espaciado para permitir cortes
\usepackage{subcaption}  % en el preámbulo
\usepackage{pgfplots}
\pgfplotsset{compat=1.18}
\usepackage{tikz}
\usepackage{xcolor}
\definecolor{myblue}{RGB}{42, 127, 179}

% Encabezado
\pagestyle{fancy}
\lhead{\textit{Ingeniería Mecánica}}
\chead{\textit{Materiales Metálicos}}
\rhead{\textit{UTN-FRVM}}

\begin{document}
\begin{titlepage}
	
	\begin{center}
		{\huge \textit{Universidad Tecnológica Nacional}}\\
        \vspace{0.5cm}
		{\LARGE \textit{Facultad Regional Villa María}}\\
		\vspace{1.5cm}
        {\LARGE{\textit{Ingeniería Mecánica - Materiales Metálicos}}}\\
		\vspace{1.5cm}
        \LARGE{\textit{Trabajo Práctico 3-06}}
	\end{center}
	
	\vfill

    \textit{Grupo DEL RÍO:}
	\begin{itemize}
		\item \textit{Abregú, Iván.}
		\item \textit{Antico, Rodrigo.}
		\item \textit{Brussa,Julián.}
		\item \textit{Cabral, Franco.}
        \item \textit{Cárdenas, Felipe.}
        \item \textit{Cardozo, Martín.}
        \item \textit{Córdoba, Nathan.}
        \item \textit{Cucco, Ramiro.}
        \item \textit{del Río, Juan.}
        \item \textit{Guerini, Nazareno.}
        \item \textit{Medina, Ivo.}
        \item \textit{Ortiz, Gastón.}
        \item \textit{Picos, Elías.}
        \item \textit{Quinteros, Lautaro.}
	\end{itemize}
    
	\textit{Docentes:}
	\begin{itemize}
		\item \textit{Dr. Lucioni, Eldo José.}
		\item \textit{Ing. Victorio Vallaro, Juan Manuel.}
	\end{itemize}
	\centering
	\today
	
\end{titlepage}

\newpage
\tableofcontents

\begin{abstract}
    A partir de la bibliografía listada a continuación, analice e investigue el contenido relacionado con el ensayo de chispas en acero (Cap. XX) a fin de adquirir efectuar una demostración práctica de sus conocimientos.
    \begin{itemize}
        \item Apraiz Barreiro, J. Aceros Especiales y Otras Aleaciones. Dossat. 5ta Edición. Madrid, 1975. Cap. XX Ensayo de Chispas (pp. 509-520) \{MM-CAD-0.0.0\}
    \end{itemize}
\end{abstract}

\section{Propósito y Fundamento del Ensayo.}

El ensayo de chispa es un método de clasificación de aceros por su composición que se realiza de forma sencilla y económica. No se utiliza para un análisis químico detallado, sino como una herramienta complementaria y de control para diferenciar y separar materiales en talleres y almacenes.


El fundamento del ensayo es la observación de las chispas que se generan al frotar el material contra una muela de esmeril a gran velocidad. El calentamiento brusco de las partículas de acero desprendidas provoca su incandescencia y la oxidación de sus elementos con el oxígeno del aire. Estas oxidaciones, especialmente las del carbono, causan explosiones en las partículas, lo que origina las figuras luminosas que se observan.

\section{Características a Observar y Zonas de la Chispa.}

Para llevar a cabo el ensayo, se recomienda trabajar en un lugar con poca iluminación y utilizar una muela de grano y dureza media. Una chispa se divide en tres zonas principales:

\begin{enumerate}
    \item La más cercana a la muela, compuesta por rayos rectilíneos con el color característico del acero.
    \item Zona intermedia donde los rayos se bifurcan y ya aparecen algunas explosiones.
    \item La zona final, donde ocurren la mayor parte de las explosiones.
\end{enumerate}


Las características clave para la identificación son la figura y el color. Se deben observar con detalle la longitud, el trazo (continuo, punteado, abultado) y la forma de las explosiones, que pueden ser estrellas, gotas, lenguas o flores.

\section{Clasificación por Elementos de Aleación.}

La forma y la intensidad de la chispa dependen de la composición del acero.
\begin{itemize}
    \item Aceros al Carbono: el grosor de los rayos, la luminosidad y la profusión de las explosiones aumentan a medida que se incrementa el porcentaje de carbono.
    \item Aceros con Molibdeno: Tienen una característica muy distintiva. En la extremidad de los rayos aparece una prolongación incandescente completamente separada, llamada \textacutedbl{}spear point\textacutedbl{}, de color rojo anaranjado.
    \item Aceros con Wolframio: Dan una chispa con rayos de color rojo oscuro, mucho menos luminosos que los de otras clases de aceros. En los aceros de alta velocidad (18\% de wolframio), los rayos son punteados y muy poco luminosos.
    \item Fundiciones: Las chispas de la fundición blanca, gris y maleable tienen características propias que permiten distinguirlas.
\end{itemize}

\section{Limitaciones del Ensayo.}

El documento señala que el estado del material (templado o recocido) tiene poca influencia en la figura de la chispa, aunque puede afectar la facilidad con la que saltan y su brillo. Sin embargo, el estado superficial del material, como la cementación o la descarburación, puede falsear los resultados.


También, se destaca que el ensayo de chispa \textbf{no es útil para determinar la presencia de níquel} en los aceros, ya que este elemento no se manifiesta con ninguna característica particular en la chispa. Para el níquel, se debe recurrir a un ensayo químico complementario.

\end{document}